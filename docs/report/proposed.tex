\chapter{Methodology}
\newpage
\section{Proposed System}

\hspace{0.6cm}Web browsers is the most commonly used medium to access websites. Docstash aims to take advantage of browser based application. The application will be available via url which the user can access via any web browser. Registered users will have access to all the information in the application.

\vspace{0.2cm}The idea of developing this application is it will ensure data integrity and security of user files. The main advantage is that it allows user to view edit and download there files from any devices. user can login with 3rd party service like Google to extend their docstash storage.


% \begin{figure}[h]
%     \begin{center}
%   \psfig{figure=./system.eps,width=6in,height=4.5in}
%     \caption{Proposed System}
%     \label{5}
%     \end{center}
%     \end{figure}

\section{Proposed Methodology}


\begin{itemize}
\item	\textbf{ANDROID (Platform/OS  for mobile application) }
Android is a mobile operating system (OS) currently developed by Google, based on the Linux kernel and designed primarily for touchscreen mobile devices such as smartphones and tablets. Android’s user interface is based on direct manipulation, using touch gestures that loosely correspond to real-world actions, such as swiping, tapping and pinching, to manipulate on-screen objects, along with a virtual keyboard for text input. In addition to touchscreen devices, Google has further developed Android TV for televisions, Android Auto for cars, and Android Wear for wrist watches, each with a specialized user interface.

\item	\textbf{SQLITE (Database for mobile application) }
SQLite is a open source SQL database that stores data to a text file on a device. Android comes in with built in SQLite database implementation. SQLite supports all the relational database features. In order to access this database, you don’t need to establish any kind of connections for it like JDBC, ODBC etc.

Features of SQLite:
\begin{itemize}
\item	•	Transactions are atomic, consistent, isolated, and durable (ACID) even after system crashes and power failures.

\item •	Zero-configuration – no setup or administration needed.

\item •	Full SQL implementation with advanced features like partial indexes and common table expressions. (Omitted features)

\item	•	A complete database is stored in a single cross-platform disk file. Great for use as an application file format.
\end{itemize}

\item	\textbf{JAVA (Technology used for mobile application) }
Android applications are developed using the Java language. As of now, that’s really your only option for native applications. Java is a very popular programming language developed by Sun Microsystems (now owned by Oracle). Developed long after C and C++, Java incorporates many of the powerful features of those powerful languages while addressing some of their drawbacks. Still, programming languages are only as powerful as their libraries. These libraries exist to help developers build applications.

Some of the Java’s important core features are:
\begin{itemize}
\item	• It’s easy to learn and understand

\item •	It’s designed to be platform-independent and secure, using virtual machines

\item •	It’s object-oriented
\end{itemize}

Android relies heavily on these Java fundamentals. The Android SDK includes many standard Java libraries (data structure libraries, math libraries, graphics libraries, networking libraries and everything else you could want) as well as special Android libraries that will help you develop awesome Android applications.


\item	\textbf{PHP (Server side technology used in web application) }
PHP is a server-side scripting language designed for web development but also used as a general-purpose programming language. As of January 2013, PHP was installed on more than 240 million websites (39 percent of those sampled) and 2.1 million web servers. Originally created by Rasmus Lerdorf in 1994, the reference implementation of PHP (powered by the Zend Engine) is now produced by The PHP Group. While PHP originally stood for Personal Home Page, it now stands for PHP: Hypertext Preprocessor, which is a recursive backronym.

\item	\textbf{HTML, CSS and JavaScript (Front end for web app) }
HTML is a mark-up language for describing web documents (web pages). HTML stands for Hyper Text Mark-up Language. A mark-up language is a set of mark-up tags. HTML documents are described by HTML tags. Each HTML tag describes different document content.

Cascading Style Sheets (CSS) is a style sheet language used for describing the presentation of a document written in a markup language. Although most often used to set the visual style of web pages and user interfaces written in HTML and XHTML

JavaScript is a high-level, dynamic, untyped, and interpreted programming language. It has been standardized in the ECMAScript language specification. Alongside HTML and CSS, it is one of the three essential technologies of World Wide Web content production; the majority of websites employ it and it is supported by all modern web browsers without plug-ins.

\end{itemize}




\textbf{MODULES OF THE PROJECT:}

The modules of the project are
\begin{itemize}

\item	Login and Registration module
`
\item	File Upload module

\item   File Encryption module

\item   Torrent Download module

\item	Chatbot module

\item	Social Media module

\item	Web Application module

\item   Backend module
\end{itemize}



\subsection{Login and Registration module:}

When the user downloads and opens the application for the first time, he will be prompted to login. In case the user does not have any previous login credentials, he can register to use the application. The registration module asks for some basic information such as username, password and flat number. Once the user has been successfully registered, he can now enter the application.

\subsection{Events/notices module:}

The events and notices module is the first module. It displays the information regarding upcoming events and notices that are active and have been uploaded by the administrator. A small description of the event will be available for the user to read to inform the user about any further steps if necessary.

\subsection{Contacts module:}

The contacts module is a useful module. It contains a list of all the important contacts that are a necessity for any housing society. It includes contact details of electricians, plumbers, laundry etc.

\subsection{Restaurant module:}
The restaurant module includes the menu of the clubhouse/society restaurant. The residents can view the whole menu and place an order directly through the applications interactive interface. The restaurant staff will receive this order along with the details of the resident who placed the order. The restaurant can then verify the order and generate the bill for the same.


\subsection{Offers module:}

This module contains useful offers that the residents can view and take advantage of. Example: 30 percent off on apparel in Amazon.

\subsection{Feedback module:}

Residents can provide their suggestions regarding various society issues and provide their feedback. They can express their opinions via the applications interface.

\subsection{Web Application module:}

The web application provides a simple interface for the society managers and secretary to

\begin{itemize}

\item	Add and update event status.

\item	Add new offers.

\item  Add and update important contacts.

\item  View feedback of various residents regarding issues.

\item	Manage restaurant menu

\end{itemize}

Along with this, the web application will have a separate login for restaurant staff to view and manage orders.

\subsection{Backend module:}

The backend runs on Apache server and contains databases for user details, events, offers, menu and feedback.


\section{System Requirements}
\begin{itemize}

\subsection{Hardware Requirements:}

\item[1.] Android Phone :


\begin{itemize}

\item[a.]	Internal Memory  - 180MB

\item[b.]	Processor –	minimum 810MHz
\end{itemize}



\item[2.] PC

\begin{itemize}

\item[a.] RAM –	512MB

\item[b.] Processor – Intel Pentium 3 1.4GHz (minimum)

\item[c.] HDD –	minimum 512MB

\end{itemize}




\subsection{Software Requirements}

Minimum software requirements are:

\begin{itemize}

\item[a.] Eclipse 4.3

\item[b.] JVM 1.6

\item[c.] Android OS – v2.3 or above

\item[d.] Windows XP,7 or above (for web app)

\item[e.] Internet Explorer 10 or above

\end{itemize}

\end{itemize}

\section{Working of the application}
\begin{itemize}
\item[1.]User can install the app using apk.
\item[2.]On opening the app, new users can proceed for registration. Existing users can directly login using their credentials. Registration and login functions are done using php requests.
\item[3.]On successfully logging in, the user can have access to all modules of the app.
\item[4.]The administrator and restaurant people manage the content of the application.
\end{itemize}


\section{Working of the Admin Panel}
\begin{itemize}
\item[1.]Administrator can access Admin panel through the url www.societyscoop.comli.com
\item[2.]He will first need to login into the Admin Panel, after which he will have access to the various functions like adding and editing events.
\item[3.]Events can be added through the add event tab and edited through the edit event tab. The event id is generated automatically. The admin will have to enter the event name and description.
\item[4.]These events are then made available to the users of the app through a php request.
\end{itemize}


    \section{Working of the Feedback module}
\begin{itemize}
\item[1.]Users of the app can fill out feedback by using a unique query id provided in the events tab.
\item[2.]The Admin can then view the feedback for different queries by using the query id in the web application.
\end{itemize}

    

    \section{Working of the Restaurant panel}
\begin{itemize}
\item[1.]The restaurant users can access the restaurant admin tab through a different url.
\item[2.]Various menu items can be added and removed by the restaurant staff.
\item[3.]Active orders can be viewed and completed orders can be sent to the completed list.
\item[4.]Users can view the menu and order directly through the application.
\end{itemize}


% \section{Diagrams}
% \begin{figure}[h]
%   \begin{center}
%   \psfig{figure=./function.jpg,width=6in,height=4.5in}
%     \caption{Working of Application}
%     \label{7}
%     \end{center}
%     \end{figure}


% \begin{figure}[h]
%   \begin{center}
%   \psfig{figure=./ADMIN.jpg,width=6in,height=3in}
%     \caption{Working of Admin Panel }
%     \label{7}
%     \end{center}
%     \end{figure}
    
%     \begin{figure}[h]
%   \begin{center}
%   \psfig{figure=./FEEDBACK.jpg,width=6in,height=2in}
%     \caption{Working of Feedback Module }
%     \label{11}
%     \end{center}
%     \end{figure}
    
% \begin{figure}[h]
%   \begin{center}
%   \psfig{figure=./Diagram3.jpg,width=6in,height=3in}
%     \caption{Working of Restaurant module}
%     \label{10}
%     \end{center}
%     \end{figure}











