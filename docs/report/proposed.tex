\chapter{Methodology}
\newpage
\section{Proposed System}

\hspace{0.6cm}Web browsers is the most commonly used medium to access websites. Docstash aims to take advantage of browser based application. The application will be available via url which the user can access via any web browser. Registered users will have access to all the information in the application.

\vspace{0.2cm}The idea of developing this application is it will ensure data integrity and security of user files. The main advantage is that it allows user to view edit and download there files from any devices. user can login with 3rd party service like Google to extend their docstash storage.


% \begin{figure}[h]
%     \begin{center}
%   \psfig{figure=./system.eps,width=6in,height=4.5in}
%     \caption{Proposed System}
%     \label{5}
%     \end{center}
%     \end{figure}

\section{Proposed Methodology}


\begin{itemize}


\item	\textbf{Nodejs (Technology used backend server application) }
Node.js is a JavaScript runtime built on Chrome's V8 JavaScript engine. Node.js uses an event-driven, non-blocking I/O model that makes it lightweight and efficient. Node.js' package ecosystem, npm, is the largest ecosystem of open source libraries in the world.

Some of the important core features of Nodejs are:
\begin{itemize}
\item	• Asynchronous and Event Driven

\item •	Single Threaded but Highly Scalable

\item •	No Buffering
\end{itemize}

This is in contrast to today's more common concurrency model where OS threads are employed. Thread-based networking is relatively inefficient and very difficult to use. Furthermore, users of Node are free from worries of dead-locking the process, since there are no locks. Almost no function in Node directly performs I/O, so the process never blocks. Because nothing blocks, scalable systems are very reasonable to develop in Node.

\item	\textbf{Docker(Software container platform) }
Docker is the world’s leading software container platform. Developers use Docker to eliminate “works on my machine” problems when collaborating on code with co-workers. Operators use Docker to run and manage apps side-by-side in isolated containers to get better compute density. Enterprises use Docker to build agile software delivery pipelines to ship new features faster, more securely and with confidence for both Linux and Windows Server apps.
\item	\textbf{MongoDB (Database for backend server) }
MongoDB is a cross-platform, document oriented database that provides, high performance, high availability, and easy scalability. MongoDB works on concept of collection and document. Classified as a NoSQL database program, MongoDB uses JSON-like documents with schemas. MongoDB is developed by MongoDB Inc. and is free and open-source, published under a combination of the GNU Affero General Public License and the Apache License.

Features of MongoDB:
\begin{itemize}
\item •Scalability (from a standalone server to distributed architectures of huge clusters): This allows us to shard our database transparently across all our shards. This increases the performance of our data processing.

\item •Aggregation: batch data processing and aggregate calculations using native MongoDB operations.

\item •Native Replication: syncing data across all the servers at the replica set.

\item •Security: authentication, authorization, etc.
\end{itemize}


\item	\textbf{React,Redux,Sass and JavaScript (Front end for web app) }

React is front end library developed by Facebook. It's used for handling view layer for web and mobile apps. ReactJS allows us to create reusable UI components. It is currently one of the most popular JavaScript libraries and it has strong foundation and large community behind it.

Redux is a predictable state container for JavaScript apps.It helps to write applications that behave consistently, run in different environments (client, server, and native), and are easy to test. On top of that, it provides a great developer experience, such as live code editing combined with a time traveling debugger.

Sass is a scripting language that is interpreted into Cascading Style Sheets 
  (CSS).Sass is completely compatible with all versions of CSS. We take this compatibility seriously, so that you can seamlessly use any available CSS libraries.The official implementation of Sass is open-source and coded in Ruby.

JavaScript is a high-level, dynamic, untyped, and interpreted programming language. It has been standardized in the ECMAScript language specification. Alongside HTML and CSS, it is one of the three essential technologies of World Wide Web content production; the majority of websites employ it and it is supported by all modern web browsers without plug-ins.

\end{itemize}




\textbf{MODULES OF THE PROJECT:}

The modules of the project are
\begin{itemize}

\item	Login and Registration module
`
\item	File Upload module

\item   File Encryption module

\item   Torrent Download module

\item	Chatbot module

\item	Social Media module

\item	Web Application module

\end{itemize}



\subsection{Login and Registration module:}

When the user downloads and opens the application for the first time, he will be prompted to login. In case the user does not have any previous login credentials, he can register to use the application. The registration module asks for some basic information such as username, password and flat number. Once the user has been successfully registered, he can now enter the application.

\subsection{File Upload module:}

The file upload module is the first module. After sucessful login user can upload any kind of data like music, videos, documents, images etc. The Metadata of this uploded file is sent to the database and then  forwarded for encrypyion. 

\subsection{File Encryption module:}

The file is divided into chunks. Each chunk is encrypted with AES 256-cbc encryption technique. And this encrypted data is saved in file system. The AES encryption algorithm defines a number of transformations that are to be performed on data stored in an array.

\subsection{Torrent Download module:}
Users can use BitTorrent protocol to download files directly to their storage. They have options of providing torrent link, magnet link or omni search. Accordingly the results will be shown to the user and after clicking on required link the torrent content will be downloaded to cloud. This displayed result of torrent download can be viewed by any user logged in to the website. 


\subsection{Chatbot module:}

Online users can chat with each other. This is implemented using real time connection in web socket.  Bot accepts users query which is processed by using Natural Language Processing (NLP) and display intended result.  


\subsection{Social Media module:}

 Users can have access to social media platform like Youtube, Twitter even if there ISP has restricted access to such websites.


\subsection{Web Application module:}

The UI is designed as a Single Page Application so that is feels like computer software as is easy to navigate as pages do not need to be refreshed regularly to get updates.






\subsection{Backend module:}

It is an nodejs server which runs inside a docker environment. It is an API based end point which serves data with the frontend. Also it scales up and down depending on the server load. 

\section{System Requirements}
\begin{itemize}

\subsection{Hardware Requirements:}




\item[1.] PC

\begin{itemize}

\item[a.] RAM –	4GB

\item[b.] Processor – Intel Pentium 3 1.4GHz (minimum)

\item[c.] HDD –	minimum 1TB

\end{itemize}




\subsection{Software Requirements}

Minimum software requirements are:

\begin{itemize}

\item[a.] Atom text editor  

\item[b.] Node.Js
\item[c.] Windows XP,7 or above (for web app)

\item[d.] Internet Explorer 10 or above


\end{itemize}

\end{itemize}

\section{Working of the application}
\begin{itemize}
\item[1.]User can access Docstash in any web browser.
\item[2.]On opening the site, new users can proceed for registration. Existing users can directly login using their credentials. Registration and login functions are done using php requests.
\item[3.]On successfully logging in, the user can have access to all modules of the site.

\end{itemize}




 


% \section{Diagrams}
% \begin{figure}[h]
%   \begin{center}
%   \psfig{figure=./activity.eps,width=6in,height=4.5in}
%     \caption{Working of Application}
%     \label{7}
%     \end{center}
%     \end{figure}


% \begin{figure}[h]
%   \begin{center}
%   \psfig{figure=./ADMIN.jpg,width=6in,height=3in}
%     \caption{Working of Admin Panel }
%     \label{7}
%     \end{center}
%     \end{figure}
    
%     \begin{figure}[h]
%   \begin{center}
%   \psfig{figure=./FEEDBACK.jpg,width=6in,height=2in}
%     \caption{Working of Feedback Module }
%     \label{11}
%     \end{center}
%     \end{figure}
    
% \begin{figure}[h]
%   \begin{center}
%   \psfig{figure=./Diagram3.jpg,width=6in,height=3in}
%     \caption{Working of Restaurant module}
%     \label{10}
%     \end{center}
%     \end{figure}