\chapter{Introduction}
\newpage

\section{Concept of Cloud Storage}

\hspace{0.6cm} Cloud Storage is a model of data storage in which the digital data is stored in logical pools, the physical storage spans multiple servers (and often locations), and the physical environment is typically owned and managed by a hosting company.

\vspace{0.2cm} Cloud storage services may be accessed through a co-located cloud computer service, a web service application programming interface (API) or by applications that utilize the API, such as cloud desktop storage, a cloud storage gateway or Web-based content management systems. 

\begin{itemize}
\item A Cloud storage has the following three main characteristics: 
    \begin{itemize}

    \item[a.] View and edit files from the cloud.
    \item[b.] Security and file encryption.
    \item[c.] Flexible storage capacity at an affordable price.
    \end{itemize}
\end{itemize}


\section{Problem Definition}

\hspace{0.6cm}  To keep up with the ever-expanding digital data of the user, storing and managing those files are difficult. The main problem that arises when dealing with storage is data security and data integrity. The user also needs to have access to this files immediately and convenience of file access is also important. Moreover, the user should have a way to share this files with others easily. Docstash attempts to bring cloud security to user’s data where they can store files reliably, proposing a new way of storage where data integrity and data confidentiality will be maintained.
\begin{itemize}
\item To develop a cloud-based application having following features :
    \begin{itemize}
    \item[a.] Ensure Storage Data Integrity.

    \item[b.] Storing use data in Encrypted format.

    \item[c.] Building user friendly interface.
    \end{itemize}
\end{itemize}


\section{Scope of project}

\hspace{0.6cm}This project aims at providing a generic platform to enhance storage management and provides integrity by encryption. Users can manage and modify their data by very user friendly interface. Each user will have storage capacity of 10 GB where they can save any file format. While the music and videos can be streamed directly via browser. All the user data will be encrypted automatically and data backup will be taken periodically.

\begin{itemize}
\item Our project is designed to achieve the following targets::
    \begin{itemize}
    \item[a.] User authentication:  It provides better user authentication through “JWT authentication”.

    \item[b.] Direct download:  It provides direct download of online content to user’s account. 

    \item[c.] Torrent download: Users can search for torrent files or can provide magnet links and it will download the files in cloud storage.
    \item[d.] Chatbot: Concept of chatbot is introduced where user can asked questions to the bot and relevant reply will be given.
    \item[e.] Social Media: All in one social media access to youtube, twitter in one place.
    \end{itemize}
\end{itemize}

\section{Relevance and Motivation of project}

\hspace{0.6cm}As more numbers users are getting access to world of internet need of data storage is required. Users need access to more amount of data daily than what they can store in their physical devices, carrying big amount storage devices is not convenient for commuters. Storing such data on cloud environment can provide easier access to any stored data. 

\vspace{0.2cm} Users want to have security and integrity to their data as data security is usual concern of data on cloud environment. Users want to make sure that their data must be kept in secure cloud environment. Traditional cloud storage providers do not allow user to expand memory which is allocated to them. However, some of the providers are trying to implement a platform so that they can fulfill user’s requirement and resolve the problem for the same.\\

\section{Organization of the report}
 The report is organized in to chapters as mentioned below:
 \begin{itemize}
 \item Chapter 2 provides with the literature survey which summarizes the problems faced by the current system.
 \item Chapter 3 focuses on the approach being followed for implementing the proposed system. It defines the functionalities of the various.

 \item Chapter 4 talks about the proposed system and its features. It also contains the various methodologies and specifications of system requirements which were in order to implement the proposed system.
 
 \item Chapter 5 enlists various diagrams to depict the flow of various modules and enhance the understanding of working of the overall system. It consists of various diagrams such as Use case diagram, UML diagram, Data flow diagram, Class diagram, Activity diagram. These diagrams highlight various modules and the way those modules connect with each other.
 
 \item Chapter 6 includes various aspects of future scope of Society Scoop and it provides  the overall outcome of the system. Several barriers and short comings of the manual system have been eliminated are discussed in this chapter.
\end{itemize}
